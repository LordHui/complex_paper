\documentclass[a4paper,10pt]{article}

\usepackage{amsmath}
\usepackage{amssymb}

\usepackage{xcolor}
\usepackage{graphicx}
\usepackage{grffile}

\usepackage{subcaption}

\usepackage{booktabs}

\usepackage{lipsum}

\newcommand{\important}[1]{\textbf{\color{red} #1}}
\newcommand{\verify}{\important{verify}}
\newcommand{\rewrite}{\important{rewrite}}

\newcommand{\real}{\mathbb{R}}
\newcommand{\cplx}{\mathbb{C}}

\renewcommand{\vec}{\overrightarrow}

\title{Sparsifying complex-valued networks}
\author{Ivan Nazarov}

\begin{document}
\maketitle

% {\color{red} \lipsum[1-3]}

\section{Related research} % (fold)
\label{sec:realted_research}

\cite{trabelsi_deep_2017} provide a set of building blocks for deep complex-valued
networks. The core of their work is structured framework for complex-valued operations
including re-im representation and linear layers (dense and convolutional) as well
as complex-valued activations and complex batch-normalization and weight initialization.

% section realted_research (end)

\section{Complex Variational Dropout} % (fold)
\label{sec:complex_varaitional_dropout}

In this section we outline and derive the variational dropout technique for complex-%
valued networks, which re-traces the evolution of dropout technique for real-valued
weights.

Variational dropout, which enables automatic relevnace detection by learning individual
per-weight dropout rates, was proposed by \cite{kingma_variational_2015} as a natural
generalization of the Gaussian Dropout \cite{srivastava_dropout_2014,wang_fast_2013},
itself -- a generalization of the Binary (Bernoulli) Dropout technique \cite{hinton_improving_2012}.

Let $W \in\real^{n\times m}$, and $b\in \real^n$ and consider the action of a linear
layer on its input $x$: $y = W x + b$ for $y\in \real^n$, $x\in \real^m$. The key idea
of Bernoulli dropout is to mask elements in the matrix $W$ of the layer with some given
probability $p$: $W = \theta \odot \xi$, there $\theta$ is the learnt weight matrix,
and is an iid Bernoulli mask $\xi \sim \mathcal{B}(\{0, \tfrac1{1-p}\}, 1-p)$. During
training, a dropout mask $\xi$ is used to compute $W$, and during evaluation the weights
are fixed to the expectation of $W$ which is $\theta$. The discussion and analysis in
\cite{kingma_variational_2015}, suggest that drawing one common dropout mask for all
data in the minibatch reduces statistical efficiency the estimator of the weights'
gradient.

Gaussian dropout works in a similar fashion, except the dropout mask $\xi$ is iid
$\mathcal{N}(1, \alpha=\tfrac{p}{1-p})$, i.e. a multiplicative Gaussian noise is
injected into the network.

% set by step we retell the story of kingmaetal2015

\cite{kingma_variational_2015} also propose the \textit{local reparameterization trick},
which improves efficiency stochastic gradient variational inference by translating
the uncertainty from the weights into local output noise of each linear layer. Without
this it would have been necessary to draw new set of random weights per each element
in a minibatch.

$$
  \mathcal{N}(\theta_{ij}, \alpha \theta^2_{ij})
  \overset{\mathcal{D}}{\sim}
  \theta_{ij} \mathcal{N}(1, \alpha)
  \,, $$

\cite{molchanov_variational_2017} propose additional step in the local reparametrization,
which further reduces the variance of the gradient estimator for each dropped out
weight. Their simple formal change of variables $(\theta, \alpha) \to (\theta, \sigma^2)$
decouples the stochastic component from the weights \rewrite:
$$
  w_{ij} = \theta_{ij} + \theta_{ij} \sqrt{\alpha} \varepsilon_{ij}
  \,\to\,
  w_{ij} = \theta_{ij} + \sigma^2_{ij} \varepsilon_{ij}
  \,, $$
with the appropriate change of variables in the Kullback-Leibler divergence.
($\alpha_{ij} = \tfrac{\sigma_{ij}^2}{\lvert \theta_{ij}\rvert^2}$).

To recap the derivation in \cite{kingma_variational_2015}, let $W$ be a random martrix
with independently distributed elements, each according to $\mathcal{N}(\theta_{ij}, \sigma^2_{ij})$.
Then the action of a linear layer on $x$ is described as 
$$
  y = W x + b = I W x + b
    = (I \otimes x^\top) \vec{W} + b
  \,, $$
where $\vec{\cdot}$ denotes flattening in lexicographic order of indices (row-major),
for which we have the identity $\vec{A B C} = (A \otimes C^\top) \vec{B}$, \cite{cookbook2012}.
% It follows from the definition of the Kronecker product of $A$ and $C^\top$ and the
% row-major order vectorization.
Next, from $\vec{W}\sim \mathcal{N}_{[n\times m]}(\vec{\theta}, \mathop{diag}\Sigma)$
we get the following distribution for the output:
$$
  y \sim \mathcal{N}_{n}
    \bigl(
      \theta x + b,
      \sum_{i} e_i e_i^\top \sum_j \sigma^2_{ij} x_j^2
    \bigr)
  \,, $$
where the variance follows from the identity $(A\otimes B)(C\otimes D) = AC\otimes BD$,
\cite{cookbook2012}:
$$
  % (I \otimes x^\top) \mathop{diag}{\vec{\Sigma}} (I \otimes x^\top)^\top
  \ldots
    % = \sum_{ij} (I \otimes x^\top) e_i \otimes e_j
    %   \sigma^2_{ij} (e_i \otimes e_j)^\top (I \otimes x^\top)^\top
    = \sum_{ij} e_i (x^\top e_j) \sigma^2_{ij} (x^\top e_j) e_i^\top
    = \sum_{i} e_i e_i^\top \sum_j \sigma^2_{ij} x_j^2
  \,, $$
with $e_i$ being the $i$-th unit vector of dimensionality conforming to the expression
is is involved with.

Criticism of \cite{gale_state_2019} implies that $\ell_0$-variational dropout,
proposed by \cite{louizos_learning_2017} and based on the concrete binary distribution
of \cite{maddison_concrete_2016}, works consistently better than the variational
Gaussian dropout.

Maths and related results \cite{pav_moments_2015,taubock_complex-valued_2012},
and \cite{karseras_caution:_nodate}

% section complex_varaitional_dropout (end)

\section{Experiments} % (fold)
\label{sec:Experiments}

I this section we will compare the complex variational dropout techniques, discussed
above, on the tasks and datasets, that were studied in prior research on complex-%
valued networks.

\subsection{Tasks and datasets} % (fold)
\label{sub:tasks_and_datasets}

\cite{trabelsi_deep_2017} studies the applicability of complex-valued networks to
image classification (CIFAR-10, CIFAR-100, reduced train of SVHN), music-transcription
(MusicNet), and Speech Spectrum prediction (TIMIT).
\begin{itemize}
  \item 
\end{itemize}


\cite{monning_evaluation_2018}
\cite{jankowski_complex-valued_1996}
\cite{amin_complex-valued_nodate}
\cite{sarroff_complex_nodate}

% subsection tasks_and_datasets (end)

\subsection{Results and Discussion} % (fold)
\label{sub:results_and_discussion}

% subsection results_and_discussion (end)

% section Experiments (end)

\clearpage

\bibliographystyle{amsplain}
\bibliography{references}
\nocite{*}

\end{document}