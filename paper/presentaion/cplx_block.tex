% https://tex.stackexchange.com/questions/176432/block-matrix-equation-with-dimensioning

% \documentclass{article}

% \usepackage{tikz}
% \usetikzlibrary{matrix,positioning,decorations.pathreplacing}

% \begin{document}


\begin{tikzpicture}[
  style1/.style={
    matrix of math nodes,
    every node/.append style={text width=#1,align=center,minimum height=2.5ex},
    nodes in empty cells,
    left delimiter=[,
    right delimiter=],
    }
  ]
  \matrix[style1=3ex] (1mat)
  {
    & & & \\
    & & & \\
    & & & \\
    & & & \\
  };
  \draw[dashed]
    (1mat-1-2.north east) -- (1mat-4-2.south east);
  \draw[dashed]
    (1mat-2-1.south west) -- (1mat-2-4.south east);
  % \draw[dashed]
  %   (1mat-6-2.south east) -- (1mat-9-2.south east);
  \node[font=\Large]
    at (1mat-1-1.south east) {$W_{11}$};
  \node[font=\Large]
    at (1mat-1-3.south east) {$-W_{21}$};
  \node[font=\Large]
    at (1mat-3-1.south east) {$ W_{21}$};
  \node[font=\Large]
    at (1mat-3-3.south east) {$W_{11}$};

  \node at ([xshift=2.5ex,yshift=-1.2pt]1mat.east) {$\times$};

  \matrix[style1=1ex,right=5ex of 1mat] (2mat)
  {
    \\
    \\
    \\
    \\
  };
  % \draw[dashed]
  %   (2mat-2-1.south west) -- (2mat-2-1.south east);
  \node[font=\Large] 
    at (2mat-1-1.south) {$x_1$};
  \node[font=\Large] 
    at (2mat-3-1.south) {$x_2$};
\end{tikzpicture}

% \end{document}
